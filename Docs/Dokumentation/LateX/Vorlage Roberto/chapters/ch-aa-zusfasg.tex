\chapter*{Kurz-Zusammenfassung}

In den letzten Jahren war ein verst�rkter Anstieg an Automatisierung in Fahrzeugen zu verzeichnen. Diese wird durch immer komplexere und intelligentere Assistenzsysteme - oder auch Advanced Driver Assistance Systems - erm�glicht. Zu diesen z�hlen beispielsweise der Notbremsassistent, durch den die Fahrsicherheit erh�ht wird. Hierzu ist eine steigende Anzahl an Sensoren und eine hohe Menge an entsprechenden Daten notwendig. Diese werden von intelligenten System- und Softwarekomponenten verarbeitet. In diesem Kontext werden vermehrt Methoden aus dem Bereich der k�nstlichen Intelligenz verwendet. 
\\
\\
Bei vollst�ndiger Automatisierung wird vom autonomen Fahren gesprochen. Hierbei werden alle Fahraufgaben vom Fahrzeug �bernommen. Das autonome Fahren ist inzwischen eine Schl�sseldisziplin des Automobilbereichs, die ein gro�es Potential aufweist. Hierbei muss die Sicherheit aller Verkehrsteilnehmer gew�hrleistet werden. Um diese sicherzustellen, m�ssen die entsprechenden Komponenten ausgiebig validiert und getestet werden. Hierzu sind eine gro�e Menge an Daten f�r verschiedenste Fahrsituationen erforderlich, um autonome Fahrzeuge auf diese vorzubereiten. F�r sicherheitskritische Fahrszenarien sind diese, aufgrund ihrer Kritikalit�t, relativ selten vorhanden. Hierbei k�nnen generative Algorithmen verwendet werden, um zeitreihen-bezogene synthetische Bewegungsdaten, f�r diese Szenarien, zu generieren.
\\
\\
Die vorliegende Bachelorarbeit behandelt die \textemph{Analyse und Gegen�berstellung von geeigneten Evaluierungstechniken zur Validierung von generativen Algorithmen im Kontext der Bereitstellung von zeitreihen-bezogenen Daten}. Hierf�r wurde, basierend auf einer umfangreichen Literaturrecherche, eine neuartige Taxonomie f�r Wahrscheinlichkeitsdichtefunktionen von generativen Algorithmen erstellt. Schlie�lich werden generative Algorithmen, auf quantitativer Weise, typischerweise validiert, indem die dazugeh�rigen Wahrscheinlichkeitsdichtefunktionen herangezogen werden.
\\
\\
Hierbei werden anwendungsrelevante Evaluierungstechniken detailliert beschrieben. Anschlie�end erfolgt, anhand ausgew�hlter Bewertungskriterien, eine Gegen�berstellung der hoch relevanten Evaluierungstechniken. Dar�ber hinaus wird eine empfohlene Evaluierungstechnik implementiert und auf synthetische Modelldaten angewandt sowie validiert.

\chapter{Schluss}
\label{sec:schluss}

Im Folgenden wird eine Zusammenfassung der Bachelorarbeit sowie zuk�nftig aufbauende Arbeiten vorgestellt. 

\section{Zusammenfassung}
\label{sec:schluss:zsm}

J�hrlich sterben etwa 1,3 Millionen Menschen bei Verkehrsunf�llen. Die h�ufigste Unfallursache ist hierbei menschliches Fehlverhalten. Aus diesem Grund besitzt das autonome Fahren das Potential, diese auf ein Minimum zu reduzieren. Hierbei werden alle Fahraufgaben vom Fahrzeug �bernommen. Diese m�ssen, um die Sicherheit aller Verkehrsteilnehmer zu gew�hrleisten, jede Situation, im dynamischen Stra�enverkehr, regelkonform bew�ltigen k�nnen. Ebenso m�ssen die dazugeh�rigen System- und Softwarekomponenten ausgiebig getestet werden. Hierzu sind eine gro�e Menge an Daten f�r verschiedenste Fahrsituationen erforderlich, um autonome Fahrzeuge auf diese vorzubereiten. F�r sicherheitskritische Fahrszenarien sind diese, aufgrund ihrer Kritikalit�t, relativ selten vorhanden. Hierbei k�nnen generative Algorithmen verwendet werden, um synthetische Bewegungsdaten f�r diese Szenarien zu generieren.
\\
\\
Die generierten Datens�tze m�ssen, um physikalisch plausible sicherheitskritische Fahrszenarien zu gew�hrleisten, anhand von Evaluierungstechniken, validiert werden. Hierbei wird typischerweise die Wahrscheinlichkeitsdichtefunktion des zugrundeliegenden Datensatzes betrachtet. Im Rahmen dieser Arbeit wurden Evaluierungstechniken f�r derartige Wahrscheinlichkeitsdichtefunktionen untersucht. Dazu wurde, auf Basis einer umfangreichen Literaturrecherche, eine Liste mit existierenden Evaluierungstechniken, erstellt. Hierbei wurden insgesamt ca. 200 Evaluierungstechniken herangezogen.
\\
\\
In einem weiteren Schritt sind die, f�r den Anwendungsfall irrelevanten, Evaluierungstechniken ausgefiltert worden. Die daraus resultierende Liste enthielt 13 hoch relevante Evaluierungstechniken. Diese konnten, auf Basis ihrer mathematischen Beschreibung, in verschiedene Gruppen eingeteilt werden. Daraus entstand eine neuartige Taxonomie von Evaluierungstechniken f�r Wahrscheinlichkeitsdichtefunktionen.
\\
\\
Die einzelnen Evaluierungstechniken wurden detailliert mathematisch beschrieben. In einem weiteren Schritt erfolgte, anhand von definierten Bewertungskriterien, f�r den zugrundeliegenden Anwendungsfall, eine Gegen�berstellung dieser Evaluierungstechniken. Hierbei konnte eine qualitative Empfehlung ausgesprochen werden. Abschlie�end wurde die empfohlene Evaluierungstechnik implementiert und anhand von synthetischen Datens�tzen validiert.

\section{Ausblick}
\label{ec:schluss:ausblick}

Die empfohlene Evaluierungstechnik wird in einer weiterf�hrenden Arbeit verwendet. Hierbei wird die entsprechende Python-Implementierung in die Pipeline eines generativen Algorithmus eingebettet. Dieser generiert synthetische Bewegungsdaten von sicherheitskritische Fahrszenarien. Die Kullback-Leibler Divergenz kann hierbei verwendet werden, um diese Datens�tze, gem�� der zugrundeliegenden Wahrscheinlichkeitsdichtefunktionen, zu validieren.
\\
\\
In diesem Zusammenhang ist ebenso eine st�rkere Pr�fung von anderen, nicht direkt empfohlenen, Evaluierungstechniken denkbar. Speziell die Wasserstein-1 Metrik oder Maximum Mean Discrepancy besitzen eine hohe Qualit�t. Hierbei m�ssen allerdings die entsprechenden Parameter bestimmt werden. Dies ist mit einem weiteren Optimierungsprozess sowie einer vertieften Literaturrecherche verbunden.
\\
\\
Im Rahmen dieser Arbeit wurden ausschlie�lich synthetische Daten analysiert. Diese weisen geringe Messungenauigkeiten auf und besitzen einen festen Wertebereich. Ebenso ist die Auswirkung von Realdaten sowie verschiedenen Wertebereichen auf die Evaluierungstechnik relevant. Schlie�lich soll diese reale sicherheitskritische Szenarien validieren k�nnen.
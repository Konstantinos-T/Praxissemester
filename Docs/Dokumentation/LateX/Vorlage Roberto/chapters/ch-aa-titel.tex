% % Neue Befehle
\newcommand{\HRule}[2]{\noindent\rule[#1]{\linewidth}{#2}} % Horiz. Linie
\newcommand{\vlinespace}[1]{\vspace*{#1\baselineskip}} % Abstand
\newcommand{\titleemph}[1]{\textbf{#1}} % Hervorheben

\begin{titlepage}
 \sffamily % Titelseite in seriefenloser Schrift
      % Logo Hochschule Esslingen
      \hfill \includegraphics[width=5cm]{fig/aa-titel/HE_Logo_4c}
      \HRule{13pt}{1pt} 
   \centering
      \Large
      \vlinespace{3}\\
      Bachelorarbeit\\
      \huge
       Analyse und Gegen�berstellung von geeigneten Evaluierungstechniken zur Validierung von generativen Algorithmen im Kontext der Bereitstellung von zeitreihen-bezogenen Daten\\
%
      \Large
      \vlinespace{2}
          im Studiengang Technische Informatik\\
          der Fakult�t Informationstechnik\\
%      
      im 7. Semester\\
%     
      \vlinespace{2}
      Roberto Corlito
%
   \vfill
   \raggedright
%   
   \large
   \titleemph{Zeitraum:} 01.09.2020 - 28.02.2020 \\ % Nur bei Abschluss-Arbeiten
%   \titleemph{Datum:} \workDatum \\ % Nur bei Studien-Arbeiten
   \titleemph{Pr�fer:} Prof. Dr.-Ing. Hermann Kull \\
   \titleemph{Zweitpr�fer:} M. Sc. Nico Schick \\ % Nur bei Abschluss-Arbeiten

% % Folgenden Abschnitt nur bei Industrie-Arbeiten darstellen
%   \vlinespace{1}
%   \HRule{13pt}{1pt} \\
%   \titleemph{Firma:} \workFirma \\
%   \titleemph{Betreuer:} \workBetreuer 
%
\end{titlepage}

\chapter{Einleitung}
\label{sec:einl}

Im Folgenden wird die Thematik dieser Bachelorarbeit n�her beschrieben. Es erfolgt eine kurze Einf�hrung in die Problemstellung und der daraus folgenden Motivation dieser Arbeit. Ebenso wird die genaue Aufgabenstellung definiert.

\section{Motivation}
\label{sec:gliederung-lit}

T�glich sterben ca. 3700 Menschen bei Verkehrsunf�llen \cite{lit:intro}. Die h�ufigste Unfallursache ist hierbei menschliches Versagen \cite{lit:af:unfall}. Durch das autonome Fahren k�nnen Verkehrsunf�lle auf ein Minimum reduziert werden. Die Technologien im Automobilbereich entwickeln sich stetig weiter. Derzeit wird verst�rkt an der Elektromobilit�t, dem autonomen Fahren und einer h�heren Vernetzung von Fahrzeugen, speziell zwischen Steuerger�ten, geforscht. Die hohe Vernetzung ist notwendig, da sich die Anzahl an Sensoren und damit an Assistenzsystemen, wie der klassische Tempomat, erh�ht. Mit der daraus folgenden h�heren Datenmenge, k�nnen intelligentere Assistenzsysteme - oder auch Advanced Driver Assistance Systems - entwickelt werden. Diese k�nnen immer komplexere Aufgaben automatisieren. Zu diesen geh�ren beispielsweise der Notbremsassistent, mit dem die Sicherheit im Stra�enverkehr erh�ht werden kann. Bei vollst�ndiger Automatisierung bzw. beim autonomen Fahren werden alle Aufgaben vom Fahrzeug �bernommen. Um autonome Fahrzeuge auf den Stra�enverkehr vorzubereiten, m�ssen Software- und Systemkomponenten effizient validiert und getestet werden. Hierbei stehen allerdings nur begrenzt Daten f�r sicherheitskritische Fahrszenarien, wie beispielsweise �berhol- oder Abbiegevorg�nge, aufgrund ihrer Kritikalit�t, zur Verf�gung. Mit Hilfe von generativen Algorithmen lassen sich synthetische sicherheitskritische Szenarien generieren. Die Herausforderung hierbei liegt in der Validierung dieser Daten.

\section{Aufgabenstellung}
\label{sec:gliederung-sys}

Es gibt verschiedene Methoden, um zeitreihen-bezogene generative Algorithmen zu validieren. In dieser Arbeit werden speziell Evaluierungstechniken von Wahrscheinlichkeitsdichtefunktionen, die einen quantitativen Ansatz haben, analysiert. Hierbei soll eine neuartige Taxonomie von Evaluierungstechniken f�r zeitreihen-bezogene generative Algorithmen erstellt werden. Dabei sind die, f�r den zugrundeliegenden Anwendungsfall, relevanten Evaluierungstechniken, mathematisch zu beschrieben. Anschlie�end sollen diese, anhand ausgew�hlter Bewertungskriterien, gegen�bergestellt werden. Das Ziel hierbei ist es, eine Empfehlung auszusprechen. Die empfohlene Evaluierungstechnik soll implementiert und anhand von synthetischen Datens�tzen, die fahrkritische Szenarien darstellen, validiert werden.
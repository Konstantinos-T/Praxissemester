\documentclass[12pt,BCOR5mm,a4paper]{article}

% % ToDo kennzeichnen
\newcommand{\workTodo}[1]{\textcolor{red}{todo: #1}}

\newcommand{\workDatum}{\today\xspace}

\usepackage{fancyhdr}
% Kopf- unf Fußzeile:
\pagestyle{fancy}
% Layout:
\setlength{\topmargin}{-20mm}										% Abstand Seitenkopf vom Rand (0mm entspricht einem Abstand von 25.4mm)
\setlength{\headheight}{15mm}										% Höhe Seitenkopfs
%\fancyheadoffset[L]{\marginparwidth}
\fancyhfoffset[LE]{6mm}
\setlength{\headsep}{15mm}											% Abstand Seitenkopf vom Seitenrumpf
\setlength{\textheight}{240mm}									% Höhe Seitenrumpf 
\setlength{\footskip}{10mm}											% Abstand unteres Ende Seitenfuss zum unteren Ende Seitenrumpf 
\setlength{\oddsidemargin}{-9.4mm}							% Abstand ungerade Seitenzahlen vom Seitenrand 
% (0mm entspricht einem Abstand vom Rand von 25.4mm)
\setlength{\evensidemargin}{-9.4mm} 						% Abstand ungerade Seitenzahlen vom Seitenrand
% (0mm entspricht einem Abstand vom Rand von 25.4mm)
\setlength{\textwidth}{184mm} 									% Breite Seitenrumpf 
% Erstzeileneinzug:
\setlength{\parindent}{0.0em} 
% Abstand zwischen zwei Absätzen:
\setlength{\parskip}{6pt plus2pt minus2pt} 			% Setzt den vertikalen Abstand zwischen Absätzen auf 6 pt
% Das Plus und Minus fügt Glue ein, d. h. TeX darf den Abstand um 4 Punkte
% vergrößern oder verkleinern, um ein gutes Layout zu erzeugen 
% (z. B. bündiger Abschluss der Seite)
% Seitenlayout auf eigene Kopf- und Kopfzeilen ändern
%\fancyhf{}													  					% Kopf- und Fußzeile löschen
\fancyhead[L]{\leftmark}
\renewcommand{\headrulewidth}{0.1pt}						% Srichstärke der Line unterhalb der Kopfzeile
\renewcommand{\footrulewidth}{0pt}						% Srichstärke der Line oberhalb der Fußzeile
\rhead{\includegraphics[width=4cm]{fig/HE_Logo.pdf}}% Logo in der Kofzeile rechts platzieren
%\cfoot{\footnotesize{\thepage}}             % Seitenzahl in der Fußzeile mittig platzieren
%%
%%% neu Defintion von Kopf- und Fu0zeilen einer leeren Seite (Titelseite, Inhaltsverzeichnis, etc.)
%\fancypagestyle{plain}{
%	\fancyhf{}
%	\fancyhead[C]{} 											% alles ausblenden
%	\renewcommand{\headrulewidth}{0.0pt} % Line unterhalb der Kopfzeile ausblenden
%	\renewcommand{\footrulewidth}{0.0pt}	% Srichstärke der Line oberhalb der Fußzeile
%	\cfoot{\footnotesize{}}             % Seitenzahl in der Fußzeile mittig platzieren
%}

% language settings
\usepackage[latin1]{inputenx}
\usepackage[ngerman]{babel}
%\usepackage[T1]{fontenc}
\usepackage{textcomp}
\usepackage{lmodern}
% wegen deutschen Umlauten
%\usepackage[utf8]{inputenc}
\usepackage[babel,german=quotes]{csquotes}

% grafik package
\usepackage{graphicx}

% pdf
\usepackage{pdfpages}

% math
\usepackage{amsmath}
\usepackage{amssymb}

% zitieren
\usepackage{cite}

% url
\usepackage{url}

% bibliographie mit natbib
\usepackage[numbers]{natbib}

% code
%\usepackage{listings}
\usepackage{xcolor}
% matlab code
\usepackage{matlab-prettifier}
\usepackage{filecontents}

% hyperref
\usepackage[hidelinks]{hyperref}

% stichwortverzeichnis
\makeindex

% figures
\usepackage{float}
\usepackage{placeins}

% wrap figures
\usepackage{wrapfig}

% multi page
\usepackage{booktabs}
\usepackage{tabularx} % tabularx nach hyperref laden
\usepackage{longtable}

% table width
\usepackage{array}
% multirow
\usepackage{multirow}

% caption
\usepackage{caption}
\usepackage{subcaption}

%\floatstyle{boxed} 
%\restylefloat{figure}

%acronyms
\usepackage{acronym}

% bibliographystyle
\bibliographystyle{plainnat}

% table caption
%\usepackage{caption}
%\captionsetup[table]{position=bottom}

% new commands
% underscore
\newcommand{\TextUnderscore}{\rule{.5em}{.4pt}}
% rename code listings -> Code Example in autoref
%\renewcommand{\lstlistingname}{Code Beispiel}% Listing -> Code Example

% rename figure -> Abbildung in autoref
\addto\extrasngerman{\def\figureautorefname{Abbildung}}
% table name
\addto{\captionsngerman}{%
	\renewcommand*{\tablename}{Tabelle}
}
% change row height of tables
\renewcommand{\arraystretch}{1.4}
% right align in longtable
%\def\equationautorefname#1#2\null{Gl.#1(#2\null)}
%\def\equationautorefname~#1\null{Gleichung~(#1)\null}
\addto\extrasngerman{\def\equationautorefname~#1\null{Gl.~(#1)\null}}
\addto\extrasngerman{\def\theoremautorefname~#1\null{Gl.~(#1)\null}}
\addto\extrasngerman{\def\AMSnameautorefname~#1\null{Gl.~(#1)\null}}

% ---- Abkuerzungen
\newcommand{\zB}{\mbox{z.\,B.}\xspace}
\newcommand{\ua}{\mbox{u.\,a.}\xspace}
\newcommand{\dah}{\mbox{d.\,h.}\xspace}
\newcommand{\Dah}{\mbox{D.\,h.}\xspace}
\newcommand{\uAe}{\mbox{u.\,�.}\xspace}

\newcolumntype{R}{>{\raggedleft\arraybackslash}p{0.65\linewidth}}
% pagestyle
%\pagestyle{option}

\newcommand{\RN}[1]{\uppercase\expandafter{\romannumeral#1}}

\definecolor{lightgray}{rgb}{.9,.9,.9}
\definecolor{darkgray}{rgb}{.4,.4,.4}
\definecolor{purple}{rgb}{0.65, 0.12, 0.82}

\lstdefinelanguage{JavaScript}{
	keywords={typeof, new, true, false, catch, function, return, null, catch, switch, var, if, in, while, do, else, case, break},
	keywordstyle=\color{blue}\bfseries,
	ndkeywords={class, export, boolean, throw, implements, import, this},
	ndkeywordstyle=\color{darkgray}\bfseries,
	identifierstyle=\color{black},
	sensitive=false,
	comment=[l]{//},
	morecomment=[s]{/*}{*/},
	commentstyle=\color{purple}\ttfamily,
	stringstyle=\color{red}\ttfamily,
	morestring=[b]',
	morestring=[b]"
}

\lstset{
	language=JavaScript,
	backgroundcolor=\color{lightgray},
	extendedchars=true,
	basicstyle=\footnotesize\ttfamily,
	showstringspaces=false,
	showspaces=false,
	numbers=left,
	numberstyle=\footnotesize,
	numbersep=9pt,
	tabsize=2,
	breaklines=true,
	showtabs=false,
	captionpos=b
}
%%% --------------------------------------------
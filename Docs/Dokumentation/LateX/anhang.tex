\section{Anhang}
\label{sec:a:intro}

\subsection{MATLAB Listings}
\label{sec:a:matlab}

\begin{lstlisting}[style=Matlab-editor,caption={MATLAB Code f�r das Modell 1},captionpos=b,label=lst:matlab1:code,language=Matlab,basicstyle=\mlttfamily,numbers=none,frame=single,escapeinside={*@}{@*}]
%% Randbedingungen / Parameter [pro Woche]

% Beute / Hasen
a1 = 0.05; % Geburtenrate: Verdopplung der Population in 20 Wochen
b1 = 0.02; % Sterberate: 2% der Hasen sterben an nat�rlichen Ursachen
c1 = 0.0006; % Fressrate der F�chse

% R�uber / F�chse
a2 = 0.0002; % Geburtenrate/Beutewahrscheinlichkeit der F�chse
b2 = 0.1; % Sterberate: F�chse verlieren pro Woche 10% Biomasse

% Anfangsbedingungen
dt = 1/7; % Zeitschritt
tmax = 51*10; % Dauer in Wochen
t = 0:dt:tmax; % Zeitvektor

% Eigene Axis f�r Phasenkurven
ax = gca;

%% Berechnung
for x20 = [25 10]

for x10 = [2500 1400 650]

	% mit Anfangswerten initialisieren
	x1(1) = x10;
	x2(1) = x20;
	
	for i = 2:length(t)
	
	% �nderung der Beutepopulation
	dx1 = a1*x1(i - 1) - b1*x1(i - 1) - c1*x2(i - 1)*x1(i - 1);
	x1(i) = x1(i - 1) + dt*dx1;
	
	% �nderung der R�uberpopulation
	dx2 = a2*x2(i - 1)*x1(i) - b2*x2(i - 1);
	x2(i) = x2(i - 1) + dt*dx2;
	
	end
	
	figure
	plot(t,x1,t,x2)
	xlabel('Zeit in Wochen','Fontweight','bold')
	ylabel('Population in Stk.','Fontweight','bold')
	legend('Anzahl Beute','Anzahl R�uber')
	set(gca,'Fontweight','bold')
	
	hold(ax,'on')
	plot(ax,x1,x2)
	hold(ax,'off')

end

end

xlabel(ax,'Anzahl Hasen in Stk.','Fontweight','bold')
ylabel(ax,'Anzahl F�chse in Stk.','Fontweight','bold')
title(ax,'Phasenkurve x_2 = f(x_1)')
set(ax,'Fontweight','bold')

figure
hold on
% Zustandsdifferentialgleichungen in anonymer Funktion speichern
f = @(t,x) [x(1)*(a1 - b1 - c1*x(2)); x(2)*(a2*x(1) - b2)];

%Phasenkurven f�r verschiedene Anfangsbedingungen bestimmen
for x20 = [25 10]

for x10 = [2500 1400 650]

	% Verlauf mit ode45 solver berechnen
	[~, xs] = ode45(f,t, [x10, x20]);
	% Phasenkurve plotten
	plot(xs(:,1), xs(:,2))

end

end

xlabel('Anzahl Hasen in Stk.','Fontweight','bold')
ylabel('Anzahl F�chse in Stk.','Fontweight','bold')
title('Phasenkurve x_2 = f(x_1)')
set(gca,'Fontweight','bold')
hold off
\end{lstlisting}

\begin{lstlisting}[style=Matlab-editor,caption={MATLAB Code f�r das Modell 2},captionpos=b,label=lst:matlab2:code,language=Matlab,basicstyle=\mlttfamily,numbers=none,frame=single,escapeinside={*@}{@*}]
%% Randbedingungen / Parameter [pro Woche]

% Beute / Hasen
a1 = 0.05; % Geburtenrate: Verdopplung der Population in 20 Wochen
b1 = 0.02; % Sterberate: 2% der Hasen sterben an nat�rlichen Ursachen
c1 = 0.0006; % Fressrate der F�chse

% R�uber / F�chse
a2 = 0.0002; % Geburtenrate/Beutewahrscheinlichkeit der F�chse
b2 = 0.1; % Sterberate: F�chse verlieren pro Woche 10% Biomasse

% Anfangsbedingungen
W = 1400; % Maximalzahl der Hasen
dt = 1/7; % Zeitschritt
tmax = 51*10; % Dauer in Wochen
t = 0:dt:tmax; % Zeitvektor

% Eigene Axis f�r Phasenkurven
ax = gca;

%% Berechnung

for x20 = [25 10]

for x10 = [2500 1400 650]

	% mit Anfangswerten initialisieren
	x1(1) = x10;
	x2(1) = x20;
	
	for i = 2:length(t)
	
	% �nderung der Beutepopulation mit logistischem Wachstum
	dx1 = 1/W*(W-x1(i - 1))*(a1 - b1)*x1(i - 1) - c1*x2(i - 1)*x1(i - 1);
	x1(i) = x1(i - 1) + dt*dx1;
	
	% �nderung der R�uberpopulation
	dx2 = (a2*x2(i - 1)*x1(i - 1) - b2*x2(i - 1));
	x2(i) = x2(i - 1) + dt*dx2;
	
	end
	
	figure
	plot(t,x1,t,x2)
	xlabel('Zeit in Wochen','Fontweight','bold')
	ylabel('Population in Stk.','Fontweight','bold')
	legend('Anzahl Beute','Anzahl R�uber')
	set(gca,'Fontweight','bold')
	
	hold(ax,'on')
	plot(ax,x1,x2)
	hold(ax,'off')

end

end

xlabel(ax,'Anzahl Hasen in Stk.','Fontweight','bold')
ylabel(ax,'Anzahl F�chse in Stk.','Fontweight','bold')
title(ax,'Phasenkurve x_2 = f(x_1)')
set(ax,'Fontweight','bold')

figure
hold on
% Zustandsdifferentialgleichungen in anonymer Funktion speichern
f = @(t,x) [x(1)*((a1 - b1)*(1/W*(W - x(1))) - c1*x(2)); x(2)*(a2*x(1) - b2)];

%Phasenkurven f�r verschiedene Anfangsbedingungen bestimmen
for x20 = [25 10]

for x10 = [2500 1400 650]

	% Verlauf mit ode45 solver berechnen
	[~, xs] = ode45(f,t, [x10, x20]);
	% Phasenkurve plotten
	plot(xs(:,1), xs(:,2))

end

end

xlabel('Anzahl Hasen in Stk.','Fontweight','bold')
ylabel('Anzahl F�chse in Stk.','Fontweight','bold')
title('Phasenkurve x_2 = f(x_1)')
set(gca,'Fontweight','bold')
hold off

\end{lstlisting}

%\subsection{User Guide: App Designer}